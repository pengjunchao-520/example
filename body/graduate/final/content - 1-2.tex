\chapter{绪论}

\section{研究背景与意义}
 
海洋是地球上最广阔的水域,它覆盖了约71\%的地球表面。海洋是我们生态系统中至关重要的一部分,它提供了许多重要的资源,包括食物、能源和药物。海洋也是一个丰富多样的生物圈,有着各种海洋生物,如鱼类、海洋哺乳动物、海洋无脊椎动物等。此外,海洋还起着调节气候、循环水分和吸收二氧化碳等重要作用。我国是一个拥有丰富海洋资源的国家,拥有近14万公里的海岸线和超过3000个海岛,有着丰富的海洋资源,主要包括渔业资源、石油和天然气资源、海洋能源资源、海洋矿产资源等。海洋资源的开发利用的前提是要了解海洋,怎么了解海洋呢?其中海底观测是了解海洋的重要手段之一。由于海底环境陆地环境相比,具有更高的压力、更低的温度和更少的光线。海底环境十分复杂,首先,海底地形十分复杂,有着各种各样的地形地貌,如海底山脉、海沟、海岭等。这些地形不仅影响着海流和海洋生物的分布,也为海洋生物提供了各种栖息地。其次,海底环境中存在着不同的海洋地质和地球化学过程,如海底火山、海底热液喷口等。这些过程会改变海底环境的化学成分和温度,影响着海洋生物的生长和繁衍。再者,海底环境也受到人类活动的影响,如海底油气开采、深海捕捞等。这些活动会对海底生态系统造成破坏,影响着海洋生物的生存。


由于海底环境的复杂性,导致我们海底观测时困难重重,对于我们的技术有着极高的挑战,现有的观测手段,主要有载人潜水器,AUV,ROV等。载人潜水器观测,对于潜水器的性能要求极高。2023年6月18日,由海洋之门营运的泰坦号潜水器在加拿大纽芬兰与拉布拉多省附近的北大西洋失踪。这艘潜水器当时正载著五名游客前往参观沉船残骸,五名人员全部遇难。AUV的海底观测有限,受限于其搭载的传感器和电池容量,需要预先设定路径,无法灵活调整。此外,成本较高,需要大量技术和设备支持。ROV依赖外部操作、受限机动性、有限的观测能力、高成本和人为干扰。这些限制了ROV在某些情况下的应用效果和性能。为了更好的应对海底复杂的环境,浙江大学陈鹰团队设计了适合海底作业的碟形AUH,并完成实验,顺利结题。针对复杂受限空间,为了让AUH更好的进行海底作业,考虑如何让AUH运动的更加敏捷是至关重要的。因此,对于AUH敏捷运动能力的提升和敏捷控制算法的改进是十分有必要的。

综上所述,AUH敏捷运动特性的研究,对于我国海底观测能力的提升,对于我国进一步了解海洋,开发海洋有着重要的意义与应用价值。

\section{国内外相关研究进展}

自从二十世纪五十年代,世界第一台AUV诞生以来,无数学者就尝试用各种方法来提高AUV的性能,而运动控制作为AUV的核心极为重要。为了使AUV适应复杂的环境,越来越多的用来提高AUV性能的控制方法应用而生。

为了解决传统的强化学习算法存在参数调节困难和学习效果不稳定的问题,
王潘宇使用了SAC算法对AUV水平面轨迹跟踪问题进行改进,采用SAC算法并对其进行了改进,包括设计合理的奖励函数、优化输入的动作空间和状态空间,
并改进经验回放机制,并通过仿真实验验证了算法的性能\cite{ref1}。
郑海斌\cite{ref2}使用PID控制算法和模糊控制算法相结合,并通过Simulink进行仿真验证,以实现水下机器人的定深定向巡航。
张冬梅\cite{ref3}用模糊PID控制和神经网络滑模控制方法对路径跟踪方案的可行性及效果进行了验证。
程健\cite{ref4}运用古尔维茨判别法对AUV的运动稳定性进行评估,仿真了AUV的直航运动和回转运动,设计了一种双环积分滑模控制器对AUV的位置进行跟踪控制。
叶梦佳\cite{ref5}采用视线导引跟踪算法和双闭环PID控制,在水平面航迹跟踪和垂直面变深任务中实现了有效的控制。
天津大学邓鲁克采用遗传算法PID对谁性爱机器人运动姿态进行控制并与PID仿真实验进行对比,得出遗传算法PID控制针对水下机器人的控制性能方面有一定提升\cite{ref6}。
杨睿提出基于水动力模型的鲁棒控制方法,利用CFD水动力模型计算补偿非线性阻尼作用所需的推进扭矩,并引入H鲁棒控制理论解决不确定性问题\cite{ref7}。
针对康达效应矢量推进器的特性,李亚鑫\cite{ref8}提出一种最优等效补偿控制方法。
该方法利用滑模控制器、神经网络和扩张状态观测器,实时估计和补偿执行器饱和和外界扰动。
通过代价函数最小化,实现推进器矢量力矩耦合特性和抑制欠驱动方向的扰动。
使用Lyapunov理论证明了方法的稳定性,并通过仿真实验验证了其轨迹跟踪和鲁棒性。 
刘璐\cite{ref9}针对AUV偏航控制系统,提出了一种鲁棒的分数阶比例积分微分 (FOPID) 控制器设计,
根据鲁棒设计规范针对参数不确定性对其他参数进行优化,仿真结果说明了所提出的控制算法具有卓越的鲁棒性和瞬态性能。


Lei通过引入快速非奇异积分终端滑模控制(FNITSM)和自适应技术,
实现速度跟踪误差的局部有限时间收敛和位置跟踪误差的局部指数收敛,提高现有自适应非奇异积分终端滑模控制方法的收敛速度和效率\cite{ref10}。
K. Eguchi使用声学传感器的海底表面估计方法、考虑车辆运动特性的目标轨迹生成方法以及使用推进器和舵的协作运动控制方法实现了AUV的低速控制\cite{ref11}。
Chengcai Wang建立了基于ADRC方法的动态控制器来跟踪参考值,进行数值模拟来分析编队控制并验证控制框架结果表明多艘AUV
可以在单线模式和V形模式之间切换和维持队形\cite{ref12}。
Vladimir Filaretov提出了一种合成控制系统的方法,该方法能够独立于 AUV 跟随器参数的变化提供所需的控制质量,实现AUV群控系统\cite{ref13}。
Hector Francisco Ortiz Villasuso将全局系统与两个单输出系统中的两个输出及其最小阶观察器解,将最小阶观测器与控制律相结合,实现AUV的深度控制\cite{ref14}。
Waseem Akram根据安装在 AUV 上的摄像机拍摄的图像自动生成部署在海底的水下管道的参考路径,通过向闭环控制系统添加外部干扰验证系统的鲁棒性\cite{ref15}。
Filaretov Vladimir提出了 AUV 空间运动控制系统的综合方法\cite{ref16},
该方法考虑了自由度、可变或未定义参数之间相互作用的存在。该控制系统由两个回路组成。第一个回路包括一个非线性调节器的组合系统,当其参数等于标称值时,该调节器提供 AUV 所需的动态特性,以及一个具有参考模型的自调节系统,该系统为未定义或可变部分提供补偿的参数。在这种情况下,选择自调节控制器的参数以减小不连续信号的可能幅度。第二个环路是非线性位置控制器,可以考虑速度环路的动态特性和 AUV 的运动学特性
为实现水下机器人在前后方向上的悬停控制,
包海默\cite{ref17}选择以黑鲈鱼作为仿生对象,
根据胸鳍的结构特点和运动方式,
设计了一种小型水下机器人仿生驱动装置,
通过伤真试验分析,得出拍动幅度和平均角速度这两个主要因素对该驱动装置的影响,
仿真试验结果表明:该胸落驱动装置可实现水下机器人在前后方向上的快速悬停 。
林银福\cite{ref18}为了探究 AUV 电机布置方式以及布置位置对 AUV 水动力性能的影响,利用CFD的方法对 AUV 进行流体计算模拟,为电机的布置形式提供了参考。
