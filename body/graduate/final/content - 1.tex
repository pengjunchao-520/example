\chapter{绪论}

\section{研究背景与意义}

\subsection{研究背景}

随着人类文明发展与社会进步,石油、天然气等自然资源消耗迅速增加,同时也随着全球经济政治变局,自然资源价格攀升。2022年以来,原油价位高位震荡:仅1月至3月,布伦特原油期货价格从79美元一桶涨至128美元一桶,为2008年全球金融危机以来最高点,7月原油期货价格为110美元一桶,仍在高位。受原油价格影响,我国2022年初至9月份共上调汽油、柴油价格14次,下调4次\cite{2022年以来主要大宗商品价格走势分析}。俄乌冲突以来,世界天然气危机日益恶化,在“北溪-1”和“北溪-2”管道被炸事件发生的后两天,有欧洲天然气价格“风向标”之称的荷兰TTF天然气期货价格从一年前的28.80欧元/兆瓦时攀升至207.19欧元/兆瓦时,而欧洲受到长期的绿能政策影响,视频、化工、机电等依赖能源的制造业受到严重影响,进而推高通货膨胀,出现经济衰退的风险及民生危机。

为了应对这一变局,全球主要国家都在进行能源结构组成优化、推动供给侧结构性改革和新旧功能转换。同时,在我国“碳达峰”和“碳中和”的“双碳”战略中,风电将起到不可替代的作用。风力发电被认为是解决全球能源需求和环境问题的有效方式之一。但目前,全球大部分风力机都安装在了陆地。陆上风电发展起步较早,发展相对成熟,但存在距离用电中心远、电网消纳能力不足、用地矛盾、视觉和噪音污染等问题。相较而言,海上风电与用电密集的沿海城市距离较近,便于电能传输与消纳;并且可以有效节约陆地资源,降低视觉和噪音污染;同时海上拥有更大的平均风速,湍流更小,利于风力机的高效稳定工作;并且海上空间更大,利于发展超大型风力机以降低成本。根据国家能源局《“十四五”可再生能源发展规划》,“十四五”期间要在东部沿海地区积极推进海上风电集群化开发。

世界海上风电论坛(WFO)在其《2022年全球海上风电报告》\cite{Council}中表示,如\autoref{fig:1-1}所示在2021年,全球海上风电装机总量93.6 GW,其中海上风电装机量21.1GW,海上风电装机量较前一年提升14.2GW。在新装机的总容量中,中国的装机量占到了世界的51\%,并且中国承诺到2030年达到排放峰值,到2060年实现碳中和(称为“30-60”目标)。为了实现这些目标,需要在第十四个五年期(2021至2025年)平均每年安装50 GW的风电,风电行业有着广阔的市场前景。其中,海上风电新装机量已经从2017年的4.5 GW增长到了2021年的21GW如\autoref{fig:1-2}所示。2021年中国的年度海上风电装机容量连续第四年居世界领先地位,新增容量近17 GW\cite{克拉克森},贡献了全球80\%的新装机量,使其累计海上风电安装量达到27.7 GW,而欧洲用了30年时间才使其总海上风电容量达到这一水平。2019年前,中国两个领先的海上风电省份江苏和广东共批准了超过26 GW的海上风电场项目。根据GWEC市场情报的全球海上风电场工程数据库,2021年初,60个中国海上风电场在建,总容量超过16 GW,其中一半以上在2020年前开始建设\cite{Ryser}。
\begin{figure}[H]
    \centering
    \begin{minipage}[t]{0.52\textwidth}
    \centering
    \includegraphics[width=7cm]{毕设图片/1-1}
    \caption{\label{fig:1-1}全球每年风机装机量增长\cite{Council}}
    \end{minipage}
    \begin{minipage}[t]{0.46\textwidth}
    \centering
    \includegraphics[width=7cm]{毕设图片/1-2}
    \caption{\label{fig:1-2}2021年五个主要国家风机装机量\cite{Council}}
    \end{minipage}
\end{figure}
\begin{figure}[htbp]
    \centering
    \includegraphics[width=5cm]{毕设图片/1-3}
    \caption{\label{fig:1-3}2017-2021年海上风电场新装机量\cite{Council}}
\end{figure}

全球海上风电市场发展前景向好,这一方面是由于2021 年举行的《联合国气候变化框架公约》第26次缔约方大会(COP26)对净零排放的承诺产生了全球性影响;另一方面是因为俄乌冲突所引发的石油、天然气等化石能源供应波动,使摆脱对俄罗斯能源依赖的紧迫性增强。预计2021-2026 年,全球海上风电装机复合年均增长率为6.3\%,2026-2031 年可达13.9\% ;新增装机容量预计将在2027年超过30 GW,到2030年有望超过50GW\cite{Ryser},其中中国海上风电预计装机量为12GW。
\begin{figure}[htbp]
    \centering
    \includegraphics[width=14cm]{毕设图片/1-4}
    \caption{\label{fig:1-4}预计全球10年内海上风电新装机量\cite{Council}}
\end{figure}

据《中国“十四五”电力发展规划研究》,我国将主要在广东、江苏、福建、浙江、山东、辽宁和广西沿海等地区开发海上风电,重点开发7个大型海上风电基地,大型基地2035年、2050年总装机规模分别达到7100万、1.32亿千瓦\cite{电力行业“十四五”发展规划研究}。随着中国上网电价补贴到期,中国创记录的海上风电建设热潮仍在继续。因此,我国已经把海上风力发电作为清洁能源发展的重要产业。

\subsection{研究意义}
海上风电开发呈现出以下特点:一是风机容量的不断增大将导致风机载荷阶梯式增长,从而导致海上风电的桩基设计长度不断增加;二是海上风电场址从近海不断走向深海,风电场水深持续增加。随着海上风电机组装机容量不断增大、离岸距离不断增加,海上风电场运维的重要性愈发凸显。根据国家发改委2019年5月发布的法规,随着国内海上风电产业的不断发展与成熟,如果在2021年底之前完全并网,2018年底之前批准的项目将获得0.85元/千瓦时的上网电价。从2022年1月1日开始,中央政府对海上风电的补贴已经终止,项目将根据电网平价计划进行支付,所以海上风电场维护成本决定着最终的电价\cite{中央财政补贴取消对海上风电项目投资收益率的影响分析}。

海上风电水下结构主要包括桩基和海缆,相对于陆上风电,海上风力机运行可靠性低、维护成本高是限制海上风电大规模开发的主要原因之一。海上风电需要考虑恶劣的海洋环境,比如盐雾腐蚀、台风破坏、海浪载荷、海流冲刷、海冰载荷等\cite{海上风电防腐蚀研究现状与前景}。海上风电桩基长期受海洋复杂环境的影响,风机倾斜、断裂、桩基不均匀沉降及腐蚀等事故时有发生,将严重影响和威胁海上风电的安全性和耐久性\autoref{fig:1-5}(a~c)\cite{大容量海上风电机组发展现状及关键技术}。海底电缆长期受到潮汐、洋流的作用,以及潜在的航运和渔业抛锚的影响,亦有受损的风险\cite{海上风电机组电气设备状态检修技术研究现状与展望}\autoref{fig:1-5}(d)。
\begin{figure}[htbp]
    \centering
    \includegraphics[width=14cm]{毕设图片/1-5}
    \caption{\label{fig:1-5}海上风电常见故障(a-b)桩基腐蚀,(c)桩基倾斜,(d)海底电缆冲刷}
\end{figure}

对海上风电水下设施进行定期巡检、探测等运维工作,可为后期健康状况评估以及处理提供依据,是保障海上风电健康发展的重要因素之一。传统的海上风电运维方式主要是在发生故障之后,派遣船将工作人员运送到故障地点,由技术人员下潜进行检测和维修\autoref{fig:1-6}。这种方式存在诸多不足。首先在于及时性较差,主要体现在风机在海上分布广泛,工作人员无法快速到达维护地点;并且海上气候条件恶劣,灾害性天气频发,工作的时间窗口常常受限;其次工作效率低下,潜水员一次下潜能够工作的范围以及能够携带的仪器设备都非常有限;再者费用昂贵,传统风电检测与维修通常需要专业的运维工程船,其成本与维修、保养费用昂贵;最后无法进行预测预警,只能在故障发生之后补救,无法防患于未然\cite{海上风电运维管理技术现状和展望}。
\begin{figure}[htbp]
    \centering
    \includegraphics[width=14cm]{毕设图片/1-6}
    \caption{\label{fig:1-6}传统海上风电巡检(a)派工程船至海上风机,(b)技术人员检修}
\end{figure}

近年来,随着人工智能技术的发展,各种类型的智能无人自主系统正在涌现。以无人机、爬行机器人、无人艇、水下机器人为代表的智能无人系统越来越多地被用于进行海上风电场巡检。如果能对海上风机水下结构进行三维重建如\autoref{fig:1-7}所示,则可以获取一个较为直观且细致的图像信息,对表面损伤和生物附着情况也有一个直观的状态获取。

\begin{figure}[htbp]
    \centering
    \includegraphics[width=6cm]{毕设图片/1-7}
    \caption{\label{fig:1-7}海上风电场水下结构三维重建}
\end{figure}

为了实现海上风电场风桩水下结构智能巡检,并提高三维重建的精度和质量,本课题着眼于海上风电水下结构巡检需求,主要研究围绕海上风电水下桩基的结构三维重建问题,提高重建精度,便于研究人员或后续算法能清晰地分辨出风桩水下结构存在的问题。从而进一步提升海上风电的运维能力,降低运维成本,提升在役风电场的效能,延长风力机工作寿命,降低发电成本,具有广阔的市场前景\cite{基于大数据的海上风电运维船舶智能管理系统}。

\section{国内外研究现状}
\subsection{水下声学三维重建}

探测深海结构物的工作情况时,图像声纳是首选的感知传感器,因为其不会被混浊、黑暗的深海条件影响。声纳发出一个触发脉冲,并在水中进行传播,碰到沙地/障碍物会反射,那么收到回声的时间越长,就说明目标物离声纳发出的距离越远。根据这一原理,研制出了图像声纳,声纳图像就是根据收到回声时间的分布产生的图像。

三种主要模式的声纳已被证明在水下的实用性\autoref{fig:1-8}。首先,侧扫声纳在海底测绘应用中取得了巨大的实用性和普遍性。其次,剖面和测深声纳提供了一个精确的窄波束,但需要很多样本才能覆盖。第三,宽口径前视多波束成像声纳提供了广阔的视场,可以灵活地从各种角度收集图像,成本远低于窄波束成像声纳。国内外基于这三种声纳,在深海做了很多三维重建的工作\cite{Assalih}。

\begin{figure}[htbp]
    \centering
    \includegraphics[width=12cm]{毕设图片/1-8}
    \caption{\label{fig:1-8}基于声纳和激光的海上风机水下结构三维重建}
\end{figure}

\subsubsection{水下声学三维重建国外研究现状}

赫瑞-瓦特大学的Thomas Guerneve等人提出了一种适用于宽孔径的图像声纳在水下三维重建的方法,能够使用低成本的成像声纳进行现场三维重建。他们发现,当环境发生变化时,声纳噪声模式几乎没有变化。首先对水箱中的声纳进行噪声模型估计,通过观测开放水域序列来估计噪声的高斯模型参数;再使用系数线性系统和反卷积算法对噪声进行近似;然后对声纳图像进行2D-3D球形拓展,以估计传感器的3D足迹;再根据多观测值优化最佳视角,并且只保留在至少一个图像中直接观察到的3D点,如\autoref{fig:1-9}(a)所示为不同波束角的声纳重建出来的效果\cite{GUERNEVE}。东京大学Yusheng Wang等人提出了一种基于图形优化的三维(3D)环境重建框架,该框架使用在水下环境中捕获的声学图像。以解决俯仰角信息的丢失问题,采用了基于声学摄像机绕声轴旋转的3D占用映射方法来生成3D局部地图。再从局部地图和图形优化方案出发,最小化相机姿态误差,构建全局地图。试验结果表明,如\autoref{fig:1-9}(b)所示,该框架能够稳健、精确地重建水下目标的密集三维模型\cite{Yusheng}。

\begin{figure}[htpb]
	\centering %将图片居中显示
	% \vspace{0.3cm}
    \subfigure[{\fangsong 宽孔径的图像声纳重建效果}\cite{GUERNEVE}] {\includegraphics[width=.45\textwidth, height = 7cm]{毕设图片/1-9-1.png}}
    \subfigure[{\fangsong 声学映射重建效果}\cite{Yusheng}] {\includegraphics[width=.45\textwidth, height = 7cm]{毕设图片/1-9-2.png}}
	\caption{不同波束角声纳重建效果图}
	\label{fig:1-9}
\end{figure}

\subsubsection{水下声学三维重建国内研究现状}

浙江大学瞿逢重等人研究了一种适用侧扫声纳图像重建水下物体三维形态的方法,侧扫声纳长挂载在船上拖曳使用,根据声纳回波估计二维强度,能获取大场景的强度图像,并根据阴影信息计算二维深度图,利用变换模型,最终重建将这两张地图合并,生成水下物体的三维点云图像,并在浙江省舟山市附近海域如\autoref{fig:1-10}所示进行试验,成功重建出海底的立方体图形如\autoref{fig:1-10}所示\cite{Jieying}。

\begin{figure}[htbp]
    \centering
    \includegraphics[width=12cm]{毕设图片/1-10}
    \caption{\label{fig:1-10}舟山试验海域重建效果\cite{Jieying}}
\end{figure}

相较于拖曳式侧扫声纳,图像声纳在使用时类似于相机,能更清晰的获取声学图像。沈阳理工大学李雪峰研究了一种声纳图像特征点三维重建的方法,采用基于马尔科夫随机场和引导滤波的声纳图像去噪、增强算法、预分割、中值滤波算法。并对声纳图像轮廓提取特征点,融合传感器参数估计,提升重建效果\cite{李雪峰}。珠海云洲智能科技股份有限公司生产的M80海洋调查无人船,于2019年搭载Coda Octopus公司实时三维声纳Echoscope对江苏海域的风电桩基及海缆进行检测,获得了桩基以及海缆状况重要数据,为进一步的维修和施工提供指导\autoref{fig:1-11}\cite{国船}。

\begin{figure}[htbp]
    \centering
    \includegraphics[width=12cm]{毕设图片/1-11}
    \caption{\label{fig:1-11}海底桩基监测情况(a)海底桩基损坏,7 m长电缆悬空,(b)升压站冲刷区,4 m悬空近10 m外露,(c)桩基有100 m外露电缆,(d)过度弯曲的电缆\cite{国船}}
\end{figure}

\subsection{光学三维重建国内外研究现状}
光学三维重建经过数十年的发展,已经取得巨大的成功。光学三维重建主要通过使用相关仪器来获取物体的二维图像数据信息,然后,再对获取的数据信息进行分析处理,利用三维重建的相关理论重建出真实环境中物体表面的轮廓信息。具有速度快,精度高等优点。

\subsubsection{空气中光学三维重建研究现状}

在空气中基于光学的三维重建主要分为基于视觉的三维重建和基于激光的三维重建。其中基于视觉的三维重建主要包括三个方面,基于SfM(Structure from Motion)\cite{SfM}的运动恢复结构三维重建算法,基于深度学习的三维重建算法,以及基于深度图的三维重建算法。

(1)基于SfM的离线三维重建

基于SfM的三维重建算法是一种离线的三维重建算法,无需输入的图片有序,算法会提取图像的特征并进行帧间位姿计算,进而对计算出位姿的图片进行立体匹配得到深度图,从而得到物体的点云。一般来说,SfM通常包括四个主要步骤:(1)稀疏点云与相机位姿计算;(2)稠密点云生成;(3)物体表面网格化重建;(4)物体表面纹理贴附。各个算法在实现时,侧重点各不相同,会更专注于其中的几个环节。

北卡罗来纳大学的Johannes L. Schonberger提出了颇有影响力的ColMap\cite{schoenberger2016sfm}\cite{schoenberger2016mvs}算法,该算法能完成计算位姿、稠密点云、网格化步骤。使用SIFT(Scale-invariant feature transform)\cite{sift}算法提取图像特征点并进行匹配,再采用光束平差法优化全局位姿的误差,然后使用Patch-Match\cite{Barnes}算法计算图像的深度图,之后合并深度图的点云。同时算法实现了泊松三维重建和德劳内三角化的表面重建算法。算法流程如\autoref{fig:1-12}所示。达姆施塔特工业大学的 Simon Fuhrmann提出了MVE\cite{Fuhrmann}算法,该算法相比较于ColMap同样能够计算位姿、稠密点云、网格化,但该算法的优势在于多尺度场景的重建,在大型场景中的分辨率较高。\autoref{fig:1-13}展示了该算法在大场景的优势。其他学者也提出了很多类似的SfM系统,华盛顿大学的Changchang Wu提出了VisualSFM\cite{Towards},该算法能够完成位姿计算和稠密点云重建,相对于ColMap,其运行速度更快,用户交互界面更加友好,但其性能和精度要稍劣于ColMap。华盛顿大学的Yasutaka Furukawa提出了CMVS-PMVS\cite{MVE}算法,该算法专注于稠密点云的计算,在相机位姿已知的情况下,利用多个视角的图像进行立体匹配,进而生成稠密的三维点云。比较适用于更密集的三维点云的应用,例如拓扑学、医学影像等领域。

\begin{figure}[htbp]
    \centering
    \includegraphics[width=14cm]{毕设图片/1-12}
    \caption{\label{fig:1-12}ColMap算法流程\cite{schoenberger2016sfm}\cite{schoenberger2016mvs}}
\end{figure}

\begin{figure}[htbp]
    \centering
    \includegraphics[width=14cm]{毕设图片/1-13}
    \caption{\label{fig:1-13}MVE算法效果图\cite{Fuhrmann}}
\end{figure}

(2)基于深度图的TSDF实时三维重建

基于深度图的三维重建方法通常能够做到实时性,通常包括计算位姿和点云融合两个模块。系统使用SLAM(Simultaneous Localization and Mapping)或者ICP(Iterative Closest Point)\cite{ICP}的方法计算出相机位姿,同时使用TSDF(Truncated Signed Distance Function)\cite{KinectFusion}方法将计算出位姿的深度图进行融合得到三维目标。

a)位姿变换估计

位姿变换估计即计算每一帧图像在拍摄时的相机位姿,不同于前述的SfM方法,这里的位姿估计是实时且连续的。SLAM方法是通过对相邻帧间进行特征点的提取、匹配、重投影误差优化计算得到精确的位姿;ICP方法则是通过相机内参将深度图转换为点云,通过三维点云配准优化计算得到位姿,基于ICP的位姿计算算法通常与重建一起出现,我们在下一节中介绍。

帝国理工大学的Andrew Davison在2007年提出了首个实时可运行的SLAM算法MonoSLAM\cite{MonoSLAM},该算法使用Harris算法检测角点,并用EKF(Extended Kalman Filter)\cite{EKF}更新优化稀疏点和相机的坐标。西班牙萨拉戈撒大学Raul Mur-Artal于2015年提出了ORB-SLAM\cite{ORBSLAM}算法,并在后续更新出ORB-SLAM2\cite{ORBSLAM2}和ORB-SLAM3\cite{ORBSLAM3}算法,该算法的是SLAM领域里程碑式的巨作。该算法有三个线程,分为前端追踪、局部建图、回环检测。前端追踪使用当前图片匹配前一帧图片和局部地图点计算初始位姿,并提取关键帧;局部建图使用前端输入的关键帧创建和删除特征点、使用重投影误差进行局部地图集束优化、剔除冗余关键帧;回环检测使用词袋模型DBow2\cite{DBow2}检测之前到达的相似场景,并与闭环候选帧进行相似变换计算位姿,再使用全局地图集束优化整体的位姿。ORB-SLAM2支持单目、双目和RGB-D图像模式,ORB-SLAM3在前面的基础上添加了IMU(Inertial Measurement Unit)、地图集模块,框架如\autoref{fig:1-14}所示。IMU使用预积分优化位姿计算,可与摄像头互相补充,在运动过快的场景下会出现图像模糊,同时图像能够修正IMU的偏移。

\begin{figure}[htbp]
    \centering
    \includegraphics[width=14cm]{毕设图片/1-14}
    \caption{\label{fig:1-14}ORB-SLAM3算法框架\cite{ORBSLAM3}}
\end{figure}

2015年苏黎世联邦理工学院的Stefan Leutenegger提出了首个融合IMU的双目SLAM算法——OKVIS\cite{OKVIS}算法。与ORB-SLAM相似,该算法也包括前端和后端,前端进行跟踪并计算初始化的位姿,后端使用IMU预积分和相机重投影进行位姿优化,并使用滑动窗口的方法优化边缘帧。2018年香港科技大学的Tong Qin提出了非常有影响力的视觉IMU融合的VINS-MONO\cite{VINSMono}算法,支持在线初始化、外参估计、回环检测功能。该算法的主要优势在于算力需求低,运行帧率高,算法框架如\autoref{fig:1-15}所示。

\begin{figure}[htbp]
    \centering
    \includegraphics[width=14cm]{毕设图片/1-15}
    \caption{\label{fig:1-15}VINS-MONO算法框架\cite{VINSMono}}
\end{figure}

b)TSDF重建

帝国理工的Newcombe等人在2011年提出KinectFusion\cite{KinectFusion}算法,其定位算法使用带法线的ICP算法,建图部分使用TSDF融合点云。可在不需要RGB图而只用深度图的情况下就能实时地建立三维模型。KinectFusion算法首次实现了基于廉价消费类相机的实时刚体重建,在当时是非常有影响力的工作,它极大的推动了实时稠密三维重建的商业化进程。后续Thomas Whelan又提出了Kintinuous\cite{Whelan},这个算法是个比较完备的三维重建系统,位姿估计结合了ICP和直接法,用GPU实现。而且Kintinuous融合了回环检测和回环优化,并且史无前例的在实时三维刚体重建中用deformation graph\cite{Sumner}做非刚体变换,根据回环优化的结果,更新点的坐标,使得回环的地方两次重建的可以对齐。Google公司Sungjoon Choi等人提出了ElasticReconstruction算法\cite{Choi},该工作提供一种从RGB-D视频重建室内场景重建的方法,其核心思想在于将场景片段的几何配准和全局优化相结合。场景片段是通过将输入RGB-D视频流分割成若干帧为一组的场景片段得到的。这种以场景片段为单位进行深度信息的融合,可以有效地去除深度图的噪声,从而获得更加准确的表面法向信息以及重建结果\cite{Qian-Yi}。

\begin{figure}[htbp]
    \centering
    \includegraphics[width=14cm]{毕设图片/1-16}
    \caption{\label{fig:1-16}KinectFusion重建效果\cite{KinectFusion}}
\end{figure}

\begin{figure}[htbp]
    \centering
    \includegraphics[width=12cm]{毕设图片/1-17}
    \caption{\label{fig:1-17}Kintinuous重建效果\cite{Whelan}}
\end{figure}

\begin{figure}[htbp]
    \centering
    \includegraphics[width=12cm]{毕设图片/1-18}
    \caption{\label{fig:1-18}ElasticReconstruction算法重建的完整公寓\cite{Choi}}
\end{figure}

Yasutaka Furukawa在提出一种基于patch的MVS\cite{Furukawa}算法。该算法的大致步骤为,首先提取输入图像的特征点,运用到方法为Harris\cite{Harris}和DoG\cite{DoG}运算。由于特征点周围的区域特征比较明显,易于与其他部分进行区分,因此下一步对已提取出的特征点进行匹配,从而获得稀疏的点云。这样直接得到的点云过于稀疏,不能满足对模型完整度的要求,因此接下来采用以特征点为基础向点的四周进行生长的方法来扩充点云,获得稠密的点云。Furukawa等人在三维重建领域进行了深入研究,收集了大量的网络上的罗马市的照片,绝大部分照片是游客、考古人员随机拍摄的无序照片。根据这些照片,重建出了罗马城的三维模型。并且应用分布式计算,采用并行优化使重建城市规模的三维模型的运行时间小于二十四小时,并且随着计算机硬件的飞速发展,算法运行时间会进一步减少。同时利用大量卫星图片,重建出了数百个城市的三维模型,应用到谷歌的下一代地图中,对我们的生活产生了极大的便利。Derek Bradley等人在文献提出了一种高鲁棒性的三维重建算法\cite{Bradley}。该方法通过自适应的改变匹配窗口的形状,解决了在物体表面与基线呈大角度时的误匹配问题。

国内也对三维重建算法进行了一定的研究,国防科大张博士提出了ROSEFusion\cite{ROSEFusion}算法,仅依靠深度图作为输入,利用随机优化求解相机位姿,实现了在快速相机移动下的室内场景稠密重建。同时该工作仅依赖深度信息,因此也可以在无光照,和变化光照的条件下使用。该工作的主要特点是:利用深度图和TSDF相容性作为代价函数,不需要提取特征点,仅依赖于深度图;提出了Particle Swarm Template(PST),利用PST可以高效的对相机位姿空间进行采样,并利用随机优化求解出相机的位姿\cite{Jiazhao}。

\begin{figure}[htbp]
    \centering
    \includegraphics[width=14cm]{毕设图片/1-19}
    \caption{\label{fig:1-19}ROSEFusion快速移动场景下的稠密重建效果\cite{ROSEFusion}}
\end{figure}

(3)基于深度学习的三维重建

基于深度学习的三维重建主要能分成两个领域,一种是采用CNN(Convolutional Neural Network)\cite{CNN}的方法对SfM进行改进,例如使用CNN直接从RGB图像中推理出姿态或者使用CNN计算场景的深度图;另一种是基于NeRF(Neural radiance fields)\cite{Nerf}的三维重建。

牛津大学的Sen Wang使用深度递归卷积神经网络(RCNNs)\cite{RCNN},提出了一种新颖的端到端单目VO的框架DeepVO\cite{DeepVO}。由于它是以端到端的方式进行训练和配置的,因此它可以直接从一系列原始的RGB图像(视频)中计算得到姿态,而无需采用任何传统VO框架中的模块。其在数据集KITTI\cite{KITTI}上测试证明其定位精度非常高。西蒙弗雷泽大学的Chengzhou Tang提出了BA-Net\cite{BA-Net}算法,这是一种根据特征度量误差明确实施多视图几何约束的网络。它通过特征度量BA来联合优化场景深度和相机运动如\autoref{fig:1-20}。整个过程是可微分的,因此可以进行端到端的训练,这样就可以从数据中学习特征,以便于从运动中构造结构。伊利诺伊大学厄巴纳香槟分校的Po-Han Huang等人提出了DeepMVS\cite{Deepmvs}算法,使用\autoref{fig:1-21}所示网络对数据集进行训练,再进行重建。能够较好地处理弱纹理区域、瘦小的结构以及反射和投影的表面,结果如\autoref{fig:1-22}所示\cite{Po-Han}。

\begin{figure}[htbp]
    \centering
    \includegraphics[width=12cm]{毕设图片/1-20}
    \caption{\label{fig:1-20}BA-Net算法框架\cite{BA-Net}}
\end{figure}

\begin{figure}[htbp]
    \centering
    \includegraphics[width=14cm]{毕设图片/1-21}
    \caption{\label{fig:1-21}DeepMVS算法网络架构\cite{Deepmvs}}
\end{figure}

\begin{figure}[htbp]
    \centering
    \includegraphics[width=10cm]{毕设图片/1-22}
    \caption{\label{fig:1-22}DeepMVS效果(效果为右)\cite{Deepmvs}}
\end{figure}

基于NeRF的三维重建由加州大学伯克利分校的Ben Mildenhall\cite{Nerf}在2020年于ECCV(European Conference on Computer Vision)上首次提出。NeRF工作的过程可以分成三部分:三维重建、渲染和训练。相对于传统的方法,该方法基于辐射场的概念,假设每个空间点都有一个辐射率,并且辐射场可以由神经网络进行建模,不依赖于传统的表面网格表示,而是直接对场景中每个点的辐射率进行建模。其核心思想是将场景中的每个点的辐射率表示为视线方向和相机参数的函数。通过在训练过程中优化神经网络参数,可以学习到场景的辐射场表示。NeRF是一种创新的神经网络架构,能够实现高质量的三维重建和渲染,具有广泛的应用前景,但其在训练和推断过程中需要大量的计算资源和时间,且对数据的采样密度和视角要求较高。

\begin{figure}[htbp]
    \centering
    \includegraphics[width=14cm]{毕设图片/1-23}
    \caption{\label{fig:1-23}NeRF流程\cite{Nerf}}
\end{figure}

在NeRF提出后,许多学者发表了相关的改进工作。2021年伦敦大学学院的SJ Garbin提出了FastNeRF\cite{Fastnerf}算法。该算法的核心方法是几何启发式分解,能够在高端消费级GPU上以200 Hz的频率渲染高保真逼真的图像。能够在空间中的每个位置处压缩缓存深度辐射图,并使用光线方向有效地查询该图以估计渲染图像中的像素值。该方法比原来的NeRF算法快3000倍,同时保持视觉质量和可扩展性。2021年康奈尔大学的Qianqian Wang提出了IBRNet\cite{Ibrnet}。该算法通过集成整合源视角图像中的信息从而获得连续空间中的颜色和密度值,克服了常规NeRF会产生伪影的问题。2021年Google的Jonathan T. Barron 提出了Mip-NeRF\cite{Mip-nerf}算法。该算法使用一种基于视锥的采样策略,实现基于NeRF的抗锯齿功能,减少了混叠伪影,并显著提高了表示精细细节的能力,同时比传统NeRF快7\%,内存占用为传统NeRF的一半。2022年卡内基梅隆大学的Kangle Deng提出了Depth-supervised\cite{Depth-supervised} NeRF算法。使用SfM产生相机位姿,生成的点云用作监督信号,有效改进了原始NeRF会对训练集过拟合的问题。

(4)基于激光三维重建研究现状

基于激光的三维重建主要是使用激光雷达直接获取环境中的物体的三维点云信息,通常使用ICP进行点云配准计算当前位姿,并通过栅格地图或者八叉树模型进行三维重建。

\begin{figure}[htbp]
    \centering
    \includegraphics[width=6cm]{毕设图片/1-24}
    \caption{\label{fig:1-24}LeGO-LOAM算法框架\cite{LeGO-LOAM}}
\end{figure}

\begin{figure}[htbp]
    \centering
    \includegraphics[width=12cm]{毕设图片/1-25}
    \caption{\label{fig:1-25}LIO-SAM算法效果\cite{LIO-SAM}}
\end{figure}

Google研究所的Wolfgang Hess在2016年提出了Cartographer\cite{Cartographer}算法。该算法的地图以子地图的形式组成,前端根据帧间匹配算法实时根据当前激光扫描结果来推测当前相对于子地图的位姿,后端使用回环检测修正各个子地图之间的位姿,该算法将地图构建过程分解为多个并行执行的子任务,因此该算法的性能和效率很高。史蒂文斯理工学院的Tixiao Shan在2018年提出的LeGO-LOAM\cite{LeGO-LOAM}算法如\autoref{fig:1-24}。相对于Cartographer算法,该算法的实时性更高,能够在嵌入式设备上实时运行;该算法利用了地面分离、点云分割和改进的L-M优化,且融合了IMU数据,在此过程中,会过滤掉可能表示不可靠特征的无值点,提升系统的鲁棒性。系统框架如\autoref{fig:1-24}所示。麻省理工学院的Tixiao Shan在2020年改进了LeGO-LOAM并提出LIO-SAM\cite{LIO-SAM}算法。该算法同样融合了IMU数据,着重于SLAM系统的全局优化,LIO-SAM算法在设计上更加注重轻量化和低计算成本,适用于资源受限的移动机器人和无人系统,提高了SLAM系统的鲁棒性和精度。算法效果如\autoref{fig:1-25}所示。

\subsubsection{水下光学三维重建研究现状}

区别于空气中的三维重建,水下的三维重建水下场景往往存在水体浑浊、纹理性弱、光照昏暗或不均匀等情况,同时水下三维重建的研究者更少。

(1)基于激光的三维重建

由于水的影响,常规的激光雷达不防水,且红外光在水下衰减的很快,因此不能在水下使用。赫罗纳大学的Albert Palomer\cite{PALOMER}等人研制了一种适用于AUV的激光扫描仪及其点云算法。首先使用EKF算法对IMU进行优化,以得到AUV自身的位姿,并获取点云的位姿。该算法包括两个步骤,首先使用平面和曲率过滤器从点云中提取关键点,以便使用更独特的关键点(即曲率更大的关键点)进行配准。然后,使用特征关联和SVD粗配准两组关键点,然后使用ICP进行精细配准。如果使用马氏距离进行的个体兼容性测试通过,则将配准结果用作多视角中的最优观察值。该算法已在一个水箱中的实际场景中进行了测试,水箱中放置了管道和阀门的结构,效果如\autoref{fig:1-27}所示,重建装置如\autoref{fig:1-26}所示。

\begin{figure}[htbp]
    \centering
    \includegraphics[width=14cm]{毕设图片/1-26}
    \caption{\label{fig:1-26}Albert Palomer等人使用激光点云重建效果\cite{PALOMER}}
\end{figure}

\begin{figure}[htbp]
    \centering
    \includegraphics[width=8cm]{毕设图片/1-27}
    \caption{\label{fig:1-27}水下激光重建装置\cite{PALOMER}}
\end{figure}

沈阳自动化研究所的Gu\cite{Gu}在2020提出采用单目线结构光的水下三维重建算法。该算法使用如\autoref{fig:1-28}所示的能旋转的水下线激光发射器和一个水下相机组成的扫描装置进行重建,通过计算在相机图像上提取激光的线所在平面与相机像素的对应射线的交点计算空间点坐标。

\begin{figure}[htbp]
    \centering
    \includegraphics[width=14cm]{毕设图片/1-28}
    \caption{\label{fig:1-28}水下结构光三维重建算法装置和效果\cite{Gu}}
\end{figure}

(2)基于视觉的三维重建

水下基于视觉的三维重建可分为大尺度三维重建和小场景三维重建两个部分介绍。大尺度三维重建与空气中三维重建方式类似,采用基于SLAM和SfM的方式进行重建;小场景三维重建依赖场景中预设好的装置,例如事先在水下架好激光扫描仪或双目相机,这种方法不能计算相机位姿,仅能对视野中的物体进行三维重建。

a)大尺度三维重建

在水下三维重建的研究中,巴黎萨克雷大学的Maxime Ferrera等人录制了一个新的数据集AQUALOC\cite{AQUALOC},用于水下航行器SLAM的研发。组成这个数据集的数据包括三个不同的环境:一个水深几米的港口,一个水深270米的考古遗址和一个水深380米的考古遗址。数据采集使用远程操作水下机器人,这些水下机器人有一个单目相机、一个低成本惯导、一个压力传感器和一个计算单元如\autoref{fig:1-29}所示。

\begin{figure}[htbp]
    \centering
    \includegraphics[width=10cm]{毕设图片/1-29}
    \caption{\label{fig:1-29}Aqualoc数据集录制器}
\end{figure}

英国Rovco\cite{营业额}公司研制了一款SubSLAM系统,该系统以ROV为载体,可以对海上风电基础设施进行实时详细建模,用户可以根据实时场景对其海底硬件进行状态评估(\autoref{fig:1-31})。该公司成立于2021年,主营水下装备重建运维,目前营业额已经达到了两千万英镑。

\begin{figure}[htpb]
	\centering %将图片居中显示
	% \vspace{0.3cm}
    \subfigure[{\fangsong Rovco研制的SubSLAM相机}] {\includegraphics[width=.45\textwidth, height = 3cm]{毕设图片/1-30.png}}
    \subfigure[{\fangsong 实时场景评估}] {\includegraphics[width=.45\textwidth, height = 3cm]{毕设图片/1-31.png}}
	\caption{英国Rovco公司水下三维重建\cite{营业额}}
	\label{fig:1-31}
\end{figure}

另有一些学者研究了水下场景中的定位问题,这也是水下大场景三维重建中的关键一环。

麦吉尔大学的Florian Shkurti\cite{SHKURTI}于2011年提出了针对水下场景的定位算法,采用视觉惯导压力计融合的方式实现,采用滤波的方式的优化位姿,并在优化过程中添加了压力与位姿的关系,但该算法是松耦合的形式,没有充分利用三个传感器的数据。上海交通大学的Ruihang Miao于2022年提出水下双目视觉融合IMU的算法UniVIO\cite{MIAO}。其在优化时融合了直接法和特征点的残差,并提供了一组水池数据集。赫瑞-瓦特大学的Elizabeth Vargas等人基于ORB-SLAM2算法研究了一种DVL作为辅助的声视觉融合的水下SLAM算法,当视觉在水下场景失效时,将启用DVL进行惯性导航,以保持SLAM算法的连续性,同时也优化了跟踪算法,以获取更多点,在水箱中验证了其精度。如\autoref{fig:1-32}所示,A为水中环境,B为原始ORB-SLAM2获取的特征点,C为改进后的算法获取的特征点,D为改进后的算法定位轨迹和DVL积分所得的轨迹,E表示漂移结果,F为数据集中的图片\cite{Vargas}。

\begin{figure}[htbp]
    \centering
    \includegraphics[width=12cm]{毕设图片/1-32}
    \caption{\label{fig:1-32}Elizabeth Vargas水下SLAM算法\cite{Vargas}}
\end{figure}

b)小场景三维重建

哈尔滨工程大学的吴云峰针对水下双目三维重建方法进行了研究\cite{吴云峰},但是其对水下折射、散射等特性的分析不够,只是将空气中的重建算法应用在水下,效果不是很理想。中国海洋大学的谢则晓、李超等人提出了两种方法,一种是基于激光扫描的水下三维重建技术\cite{解则晓},另一种是用平行双目相机进行水下三维重建\cite{李超}。哈尔滨工业大学的李洪生对水下图像复原技术\cite{李洪生}、标定技术进行了深入的研究,运用逆滤波、维纳滤波等方法进行图像的复原,并且建立了水下相机成像的非线性模型。哈工大魏景阳研究了一种双目立体视觉水下三维重建的算法,将空气中的相机进行标定,并对像素偏移误差进行补偿,再应用空气中的三维重建算法,效果如\autoref{fig:1-33}所示\cite{魏景阳}。

\begin{figure}[htbp]
    \centering
    \includegraphics[width=6cm]{毕设图片/1-33}
    \caption{\label{fig:1-33}马头模型3D偏差示意图}
\end{figure}

国内的青岛市光电工程技术研究院提出了一种RGB激光扫描进行三维重建的方法\cite{董会}。如\autoref{fig:1-34}(a)所示,RGB三基色激光器和CCD相机安装在直线滑轨上,可由步进电机带动旋转。RGB三基色的激光作为主动光源,经过分光镜和反光镜形成白色线结构光,当CCD相机标定后,根据水下衰减曲线得到补偿值,再调节衰减片与激光器输出光强。扫描完成后,对扫描记录的图像进行坐标变换,得到图像的二维矩阵,沿扫描方向上叠加,可以得到三维矩阵,即可得到三维重建图像,完成目标场景的三维信息重建,效果如\autoref{fig:1-34}(b)所示。


\begin{figure}[htpb]
	\centering %将图片居中显示
	% \vspace{0.3cm}
    \subfigure[{\fangsong 线结构光扫描示意图}] {\includegraphics[width=.45\textwidth, height = 4cm]{毕设图片/1-34-1.png}}
    \subfigure[{\fangsong 颜色补偿效果}] {\includegraphics[width=.45\textwidth, height = 4cm]{毕设图片/1-34-2.png}}
	\caption{RGB激光扫描装置}
	\label{fig:1-34}
\end{figure}

\subsection{水下重建图像预处理国内外研究现状}
在视觉水下三维重建的工作中,由于相机在水下场景中作业时,光线会穿过水-玻璃-空气投影到相机光芯,而穿越不同介质会造成光的折射,折射现象导致水下三维重建更为复杂。国内外许多学者也进行了这方面的研究,包括建立光路在水下场景的折射模型、标定水下相机的折射参数、设计图像复原算法。

\subsubsection{相机折射模型}

一些研究者们直接将空气相机的SVP(Single View Point)模型应用于水下。加州大学的Treibitz\cite{Treibitz}在2011年证明了SVP模型不适于水下的环境。皇家墨尔本理工大学的Shortis\cite{Shortis}在2015年 对比了SVP模型和折射模型,也验证了SVP模型结果不太理想。

一些研究者们通过设计补偿模型的方法优化折射误差。上海交通大学的Longxiang Huang\cite{Huang}在2015年使用虚拟光芯坐标的简化折射模型,减小折射误差,但没有考虑玻璃的法线与相机Z轴不平行的真实情况。不莱梅雅各布斯大学的Łuczyński\cite{Łuczyński}在2017年提出了Pinax模型,找到最合适的虚拟光芯位置,该方法在已知玻璃厚度和舱体中空气厚度的情况下仅需要在空气中进行标定,但这种方法仅当物体点处在其考虑的一定距离内有效,在大场景不具有普适性。这种方法在水下目标识别领域是有效的,因为其不用考虑物体表面点的坐标信息,而在三维重建领域,对像素坐标点的精度要求非常高。

另一些研究者们根据水下光线折射的物理性质研究了光路折射模型,三菱电机研究所的Agrawal\cite{Agrawal}在2012年通过构建折射平面的方法将三维空间的折射问题投影到二维平面上,从而构建出空间点到相机光芯的线性投影方程,并使用线性变换算法DLT和RANSAC\cite{RANSAC}方法求解法线,再根据多层折射模型求解每一个折射层的厚度,但这种方法在计算时,需要解一个12元的线性方程,且法线是由法线和旋转矩阵的乘积矩阵分解求得的,实际使用时误差比较大。德国基尔大学的Jordt-Sedlazeck\cite{Jordt}于2012年在Agrawal的方法基础上添加了合成分析方法,最小化渲染和测量的灰度值之间的平方差的总和。

另有一些研究者们只考虑了单层光路折射,忽略玻璃面的影响。特里凡得琅工程学院的Parvathi\cite{Parvathi}在2019年计算了水到空气的折射模型,并计算了双视角折射模型。

\subsubsection{水下相机标定}

在水下相机标定领域,一些学者设计了不考虑玻璃介质的单层折射模型的标定方法。隆德大学的Haner\cite{Haner}在2015年计算了单次折射模型下的折射定律下的模糊解问题,在已知法线的情况下能够计算出相机的在世界坐标系下的绝对位姿。上一节中介绍的上海交大学者提出的的UniVIO\cite{MIAO}忽略了玻璃对光线的影响,这种假设仅当玻璃很薄时近似成立,而海上重建的工作环境常是深远海,为了适应强大的水压,玻璃板通常会设计的比较厚实 。

一些研究者考虑了玻璃面在内的双层折射模型,华盛顿大学Henrion\cite{Henrion}在2015年将标定板换成一个有50个已知的校准点的打印件,最小化校准点与其对应的光线距离,从而计算内外参数,但这种方法受打印件精度影响,且需要知道校准点在相机坐标系下的坐标。沈阳自动化研究所的Gu\cite{MedUCC}在2021年提出了一种介质驱动的水下摄像机标定方法MedUCC,如\autoref{fig:1-35}所示,该算法将相机和标定板固定,分别在无玻璃无水、有玻璃无水和有玻璃有水的情况下分别拍摄标定板,该方法拍摄过程复杂,一些相机无法在固定的情况下拆卸玻璃,不具有通用性。

\begin{figure}[htbp]
    \centering
    \includegraphics[width=14cm]{毕设图片/1-35}
    \caption{\label{fig:1-35}MedUCC算法框架}
\end{figure}

一些研究者们结合光路折射模型和色散效应提出了水下相机标定方法,阿尔伯塔大学的Chen\cite{closed}在2017年一种标定方法,该方法 基于不同频率的光线的折射率不同的原理,相机会感受到物体发出的RGB三种颜色的光的强度,三种频率的光会发生色散,而其共用法线和到玻璃面的距离,从而能够解出相机参数,但这种方法需要高精度、高分辨率相机,而这种图片数据量过大,不适用于三维重建,普通相机难以区别出三种颜色细微的色散效果。

一些研究者们也研究了水下双目相机的标定方法,上海大学的Zhang\cite{Zhang}和上海应用数学与力学研究所的Su\cite{Optics}分别在2018年和2021年将Agrawal的方法应用于水下立体相机,对两个相机分别标定,采用非线性优化的方法计算法线、到玻璃面的距离和旋转平移矩阵,但由于水下相机舱体体积小导致的视场角限制,Su所使用的的在玻璃上贴标记的方法无法使用。奥尔堡大学的Pedersen\cite{Pedersen}在2018年提出了水下双目相机三角化计算空间点坐标的方法,并在水箱中做了试验,但该方法只处理了单个点的坐标,并考虑了单层折射。

\subsubsection{水下图像折射复原}

关于将水下照片恢复为航空图像的研究很少。Łuczyński\cite{Łuczyński}使用Pinax近似模型恢复图像。如\autoref{fig:1-36}所示,该算法将水中的折射模型使用虚拟光芯位置的近似方法替代,但该方法存在精度损失并在大场景 3D 重建中会产生累积错误。

\begin{figure}[htbp]
    \centering
    \includegraphics[width=14cm]{毕设图片/1-36}
    \caption{\label{fig:1-36}Pinax图像恢复算法原理和效果}
\end{figure}

\subsection{存在的问题}

总体来看我国在使用三维重建技术对海上风电场进行智能运维的领域起点低,起步较晚,研究较少。我国对海上风电场的监测以传统手段为主,主要采用卫星活航天器遥感监测,船舶在航监测,难以监测到风机水下基础结构的运行状态和结构损伤,同时在航监测又具有一定的危险性,监测窗口期受环境影响较大,随着海上风电的体量逐年增加,传统监测方法已经无法保障海上风电场的安全稳定运行。

(1)声纳图像方法虽然不受深海环境的干扰,能建立出物体的轮廓,无法建立出详细的纹理,并且其价格高昂,同时又受西方国家打压,国内难以获取高精度的设备,其成像精度将会被进一步压缩。而水下风桩的故障,例如锈蚀和生物附着的损伤往往体现在其表面的颜色上,只能使用深海相机等图像设备对其进行重建。图像声纳难以用在海上风机水下桩基的巡检任务中。

(2)在基于激光的水下三维重建中,空气中的激光雷达由于不防水且其测量点云所基于的红外光在水下衰减的很快,因此不适用于水下三维重建;而一些学者自行搭建的水下激光扫描装置存在成本高昂,扫描速度慢,没有颜色信息等问题。

(3)在视觉水下三维重建领域,一些研究者通常只研究了一个视角的小场景三维重建,例如在水中搭建标定好的扫描仪和单目线结构光装置,这种装置和算法不适用在海上风电场这种大尺度场景中。大尺度重建需估计相机的运动,但在特征点少的水下环境中,常规的特征点匹配法和直接会受到水下物质散射的影响,导致跟踪失败,而双目光流法在复杂的环境中具有更高的鲁棒性;传统的SfM和深度学习三维重建算法是一种离线算法,在海上作业完成后才能计算重建效果;另一方面这种算法是将相机图片作为单目相机处理,难以获取绝对位姿,仅能获取图片的相对位姿,同时这种方式不能融合IMU和深度计等传感器,在水下场景中的鲁棒性难以提升。且在海上作业时,窗口期宝贵,若拍摄时不能实时看到重建效果,会导致点云残缺而不能及时发现。现有的基于TSDF三维重建的算法均使用深度图计算位姿,而在水下深度图精度有限,难以达到重建的位姿要求,且现有的TSDF重建算法只能计算出三维点云,并没有后处理步骤,所生成的三维体表现模糊,并不利于故障诊断。现有的水下多传感器融合定位SLAM算法没有考虑深度计相对于体坐标系的外部坐标变换参数,存在系统误差。因此需要研究基于光流跟踪的多传感器融合的双目定位算法,与实时重建算法,并在后续过程中完成网格化和纹理贴附,以重建出逼真的场景,能够在巡检和数字孪生等场景中发挥更大的作用。

(4)研究者们推导出了相机的光路折射模型,但忽略了折射时的玻璃与相机安装的角度、玻璃与相机距离引起的折射误差,影响成像精度。且现有的水下相机标定算法一方面实施过程较为复杂且精度不高,需要对相机进行反复拆装;另一方面将模型过度简化,仅研究了光线从水中到相机空气的折射,没有考虑相机舱体玻璃板对光线的折射,导致模型污垢精确。所以需要对光路进行矫正并研究出适用于水下相机双层折射模型的标定算法。

\section{研究内容}
本课题着眼于海上风电水下结构巡检需求,旨在研制一种基于双目视觉的多传感器融合的水下三维重建技术,实现对海上风电水下结构物表面进行三维重建。组装一个适用于海上巡检的水下双目相机舱体,安装水下双目相机、IMU和深度计,事先标定好水下双目相机。在巡检时将数据发送到岸上主机进行实时三维重建,移动相机获取完整的点云,拍摄完风机水下结构后,对点云进行网格化和纹理贴附操作,保存重建结果。便于专家进行识别与后续数字孪生处理,以进一步提升海上风电水下结构巡检的无人化与智能化水平。

\begin{figure}[htbp]
    \centering
    \includegraphics[width=10cm]{毕设图片/1-37}
    \caption{\label{fig:1-37}海上风电场\cite{海上风电场}}
\end{figure}

本工作包括水下双目相机标定算法设计,水下图像折射复原算法设计,多传感器融合水下目标实时定位跟踪,水下实时重建与网格化和纹理贴附算法设计,以及水下三维重建系统设计与风桩桩基巡检试验研究四个研究内容。

(1) 水下双目相机标定算法设计。本文通过分析自研的水下双目相机,对成像模型建模,提出了一种基于成像坐标约束的水下双目相机折射标定算法。本文的方法先在相机安装前使用传统标定方法计算出相机内参和双目相机相对位姿;再使用本文提出的非线性优化的方法在水下进行标定,计算折射参数,折射参数包括玻璃板在相机坐标系下的距离、法线角度和厚度,使得双目折射三角化后的点与标准标定板的坐标误差最小,本节给出了该算法的仿真试验并用现有算法进行对比分析。相较于现有的方法,本文提出的方法仅需在水下任意角度拍摄标定板,无需反复拆装镜头;且算法鲁棒性好精度高。

(2) 水下折射复原算法设计。本节利用上一节所提出水下双目相机标定算法计算出的折射参数恢复无水状态下的RGB图和深度图。首先使用现有算法计算双目相机的视差图,推导出双目相机像素点与对应的空间三维点的数学关系,得到水下图像对应的深度图,再通过针孔模型反投影到相机像素平面,即可得到在无水状态下的左右相机的RGB图和深度图。本文介绍了复原算法的GPU实现,可达到实时的效果。

(3) 水下多传感器融合实时定位算法设计。本节使用上一节获取的无水状态的RGB图和深度图并与IMU深度计数据融合进行水下定位。算法分为前端、建图、后端优化和回环检测四个线程。算法前端使用图像特征点进行光流法匹配计算初始位姿,并计算IMU的预积分;建图线程使用左右相机和前后帧对特征点进行三角化,计算出空间点的三维坐标;后端并行进行优化,将三个传感器变换到IMU坐标系下,使用视觉重投影误差、IMU预积分残差和深度计的高度残差联合优化;回环检测线程会计算图像的词袋模型,当相机再次运动到这一场景时会进行闭环优化,优化整个地图中的相机位姿和空间点坐标。最后将优化后的位姿导入到三维重建线程。

(4) 水下三维实时重建系统设计。本节利用上一节发送来的相机位姿、RGB图和深度图进行实时三维重建,生成稠密点云,并在后处理中对点云进行网格化和贴图操作。本文使用GPU运行TSDF算法对深度图进行融合,实时显示稠密点云重建的结果,并在使用GPU计算时记录每一个TSDF体素的观测信息。在作业完成后导出三维点云,使用德劳内三角化进行网格化,再使用光度一致性对网格进行细化,最后使用基于马尔科夫随机场的置信度传播算法和泊松融合算法对网格进行纹理贴图,完成三维重建。

(5) 水下三维重建试验与分析。本节使用公开数据集和实测数据对前文所提出的算法进行理论验证和量化分析。本文为了使装置能适应水下的工作环境,搭建了一套巡检ROV系统,ROV能够携带探测模块在水下运行,探测模块集成双目立体相机系统、IMU和深度计。首先使用我们搭建的水下双目相机验证我们所提出的标定算法和图像折射复原算法,再在公开数据中验证我们提出的多传感器融合定位算法的精度,最后在实际水池中使用我们的重建算法对海上风电水下桩基模型进行三维重建。

\section{章节安排}
本文第一章首先介绍研究背景和研究意义,并分析海上风电机组的工作环境、常见故障和现有巡检方式的不足之处,从而引出本文的研究内容。第二章介绍了水下双目相机成像模型,并提出了基于非线性优化的水下双目相机标定算法和折射复原算法,通过仿真对比现有标定算法证明算法优势。第三章介绍了基于双目相机、IMU和深度计的多传感器融合位姿计算算法,推导出了三种传感器的紧耦合残差计算公式和雅克比公式。第四章介绍了实时三维重建算法,给出了基于GPU的TSDF实时重建以及后续网格化和纹理贴附的原理和实现方法。第五章为试验与分析,搭建了一套适用于海上风电场巡检的ROV和双目相机,并给出了在实测数据和公开数据集上本文提出了每一个算法的测试结果。第六章总结全文的研究工作和内容,总结研究的不足之处并对未来进行了展望。全文结构安排如\autoref{fig:1-38}所示。

\begin{figure}[htbp]
    \centering
    \includegraphics[width=14.5cm]{毕设图片/1-38}
    \caption{\label{fig:1-38}全文结构安排}
\end{figure}