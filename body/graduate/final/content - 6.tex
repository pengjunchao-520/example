

%第六章
\chapter{总结与展望}

\section{研究工作总结}

本文针对海上风机水下结构巡检问题开展研究,旨在研究基于双目视觉的水下三维重建算法与系统,从而提升海上风电水下结构的精确感知能力。具体而言,实现了水下结构视觉三维重建完整流程,提出了适用于水下双目相机的标定算法、水下图像折射复原算法、实时水下“视觉-惯性-深度”紧耦合位姿优化算法和水下实时稠密三维点云融合算法,并实现了点云后处理网格化和纹理贴附,且关键算法均完成仿真及试验验证与对比分析。全文主要工作总结如下:

(1)提出了一种水下双目相机折射标定算法和图像复原算法。建立了水下双目相机的折射成像模型,提出了一种基于成像坐标约束的水下双目相机折射标定算法。通过仿真和试验对比了本文算法和Kong的水下双目相机标定算法、空气针孔模型算法、水下吸收针孔模型算法,本文算法水下坐标计算精度可达98.46\%,证明了其在标定结果上具有更高的精度。本文还针对水下三维重建中的折射现象设计了水下图像复原算法,利用视差图和成像模型计算出了整个图像中像素对应三维点,并通过针孔模型反投影回像素平面,供后续算法使用。通过水池试验对比了Pinax模型,证明本文算法能实现像素级的图像复原,误差为Pinax模型的1/3左右。

(2)提出了一种基于“双目视觉-惯性-深度”多传感器融合的水下定位定姿算法。使用经过折射复原的双目相机左右图像和深度图、IMU、深度计数据计算水下位姿,通过改进OV2SLAM算法,使用Fast算法提取特征点、光流法进行图像特征点匹配,使用IMU积分和PnP算法预测位姿,使用体坐标系下的双目相机重投影误差、IMU预积分误差、深度计的深度值误差计算系统定位误差,实现了三个传感器联合初始化,以及基于共视关键帧和滑动窗口法的紧耦合位姿优化。本文分别使用空气双目IMU数据集EuRoC、水下单目IMU深度计数据集Aqualoc、水下双目IMU深度计数据集Cave进行数据集试验,在Cave数据集中的完成度达到了100\%,ATE结果为0.32 m。结果表明本文RRVIP-SLAM算法与经典VIO算法VINS-Fusion、ORB-SLAM3和改进前的OV2SLAM相比较,在空气中有相似精度,在水下场景中有更好的鲁棒性和定位精度。

(3)实现了基于TSDF的水下实时三维重建算法和离线三维点云后处理算法。本文在实现TSDF时使用了CPU数据交互和GPU权重体素计算的方式实时对场景进行三维重建,在体素中提取TSDF值的零点获取点云及对应的颜色信息。在后处理阶段使用德劳内三角剖分、表面网格提取、网格细化和纹理贴附的方法实现精细重建。在此基础上,使用ICL-NUIM数据集进行测试,计算得到云坐标误差为0.0074 m,并通过对比实时三维重建算法ROSE-Fusion和Bundle-Fusion,证明了本文RRVIP-SLAM算法在点云坐标计算精度和视觉效果上均有提升。

(4)完成了海上风机水下结构水池模型试验与对比分析。设计研制了一套适用于海上风电水下结构巡检的双目相机,搭建了海上风机导管架水池试验模型,完成了水下相机标定、图像复原、位姿计算、三维点云融合以及纹理贴附等关键算法的仿真和试验验证。结果表明,本文RRVIP-SLAM 算法在海上风电水下结构模型上的三维重建精度均达到毫米级,而主流离线三维重建算法ColMap 和实时三维重建算法ROSE-Fusion存在跟踪丢失等问题,导致三维重建失败,从而对比证明了本文所提算法在水下三维重建场景中的有效性。

\section{主要创新点}
本文的创新点主要包括:

(1)提出了一种基于成像坐标约束的水下双目相机折射标定算法,构建了水下双目相机成像模型,设计了基于棋盘格坐标拟合的优化目标函数,并通过仿真和试验证明了本文算法的有效性和高精度。

(2)提出了一种基于“双目视觉-惯性-深度”多传感器数据融合的水下定位定姿算法,设计了基于共视帧关键帧和滑动窗口的紧耦合优化算法,并分别在公开数据集和水池试验中证明了本文算法的测量精度和鲁棒性。

% (1)针对水下三维视觉重建问题设计了完整的三维重建系统,包括水下图像预处理、水下视觉-IMU-压力计紧耦合定位系统、实时TSDF三维点云融合、离线后处理。创新性地贯通了TSDF和点云后处理步骤,在实现高精度重建的同时又能获取良好的表面纹理。

% (2)针对水下视觉中的光线折射问题,提出了基于成像坐标约束的水下双目相机折射标定算法和水下图像折射复原算法。通过构建双目成像模型,计算在相机坐标系下的标定板坐标,通过计算出的坐标构建坐标系,并于实际坐标系做拟合,解决了在水下相机标定时难以获取标定板到相机的变换矩阵的问题,并能计算出准确的标定参数。通过相机成像模型和水下视差图计算水中物体的三维坐标点,并反投影至像素平面,以计算折射复原后的图像。



\section{研究展望}

本文针对海上风机水下结构的三维重建问题,旨在研究基于双目视觉的水下三维重建算法与系统,从而提高海上风电水下结构的精准感知能力。然而,受限于个人精力和能力限制,本课题的研究依然有很多不足之处。立足于已有的研究,后续可以从以下方面进行深入:

(1)在水下图像预处理方面,区别于空气中的三维重建,水下环境会对不同波长的光造成不同程度的衰减,后续工作能设计算法实现水下图像颜色复原。同时为了减少重建时背景对TSDF融合的干扰,后续工作能设计水下背景分割算法,将水下结构物与背景中的水分割开。

(2)在水下SLAM定位方面,后续可以开发基于九轴IMU的定位算法,相比与常用的六轴IMU,九轴IMU通过地磁能够测出三个方向角的绝对值,通过紧耦合算法融合优化能使系统在水下场景中具有更高的鲁棒性。后续也可以开发基于视觉-IMU-深度计-图像声纳融合算法,图像声纳在水下环境中能测出大范围高精度的点云,例如使用声纳点云ICP计算残差,实现紧耦合位姿计算。

(3)在未来可进行海上风电场海试试验,测试本文RRVIP-SLAM算法在实际风电场场景下的实时定位效果、重建效果与本文后处理的纹理效果,同时测试算法在不同海况下的稳定性,并根据测试结果对算法反馈优化。

