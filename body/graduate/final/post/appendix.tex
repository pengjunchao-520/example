\cleardoublepage
{
    \chapternonum{附录}

    \appendixsecmajornumbering

    \section{附录1 现场风电机组SCADA系统详细参数 }

    \begin{table}[H]
        \caption{\label{tab:6-1}某风电场SCADA系统1-71监测变量}
        \small %此处写字体大小控制命令
        \centering%把表居中
        \renewcommand{\arraystretch}{1.4}
        \setlength{\tabcolsep}{4mm}{
        \scalebox{1.0}{
        \begin{tabular}{cccc}
        \toprule%第一道横线
        序号    & 监测变量           & 序号      & 监测变量         \\
        \midrule%第二道横线 
1     & 机舱振动x方向      & 35    & 偏航速度         \\
2     & 机舱振动y方向      & 36    & 对风偏差         \\
3     & 机舱振动最大偏移值    & 37    & 变桨控制服务设定值    \\
4     & 机舱振动有效值      & 38-39 & 齿轮箱轴1-2温度    \\
5-7   & 风速1-2        & 40    & 齿轮箱入口油温      \\
8-9   & 风向1-2        & 42    & 齿轮箱冷却水温度     \\
10    & 25秒风向        & 43    & 润滑油滤网入口压力    \\
11    & 60秒风向        & 44    & 润滑油滤网出口压力    \\
12-14 & 叶轮转速         & 45-46 & 发电机绕组温度U1-U2 \\
15    & 叶轮转速限制       & 47-48 & 发电机绕组温度V1-V2 \\
16    & 叶轮位置         & 49-50 & 发电机绕组温度W1-W2 \\
17    & 转子转速         & 51    & 发电机转速        \\
18    & 主液压系统压力      & 52-53 & 发电机轴承a-b温度   \\
19    & 叶轮刹车系统液压     & 54    & 发电机冷却风温度     \\
20    & 液压油温度        & 55    & 发电机滑环温度      \\
21    & 3秒平均风速       & 56    & 环境温度         \\
22    & 10秒平均风速      & 57    & 机舱温度         \\
23    & 30秒平均风速      & 58-60 & U1-U3项绕组电压   \\
24    & 60秒平均风速      & 61-63 & U1-U3项绕组电流   \\
25    & 10分钟平均风速     & 64    & 发电量          \\
26-27 & 60秒平均风向1-2   & 65    & 电网功率         \\
28-29 & 25秒平均风向1-2   & 66    & 电网无功         \\
30    & 10分钟变桨叶片平均位置 & 67    & 功率因数         \\
31    & 10秒变桨叶片平均位置  & 68    & 电网频率         \\
32    & 1秒变桨叶片平均位置   & 69    & 塔底柜温度        \\
33    & 变压器温度        & 70    & 塔顶柜温度        \\
34    & 偏航位置         & 71    & 主轴叶轮侧温度      \\
        \bottomrule%第三道横线
        \end{tabular}}}
        \end{table}
    
        \begin{table}[H]
            \caption{\label{tab:6-2}某风电场SCADA系统72-191监测变量}
            \small %此处写字体大小控制命令
            \centering%把表居中
            \renewcommand{\arraystretch}{1.4}
            \setlength{\tabcolsep}{4mm}{
            \scalebox{1.0}{
            \begin{tabular}{cccc}
            \toprule%第一道横线
            序号    & 监测变量           & 序号      & 监测变量         \\
            \midrule%第二道横线 
            72      & 主轴齿轮箱侧温度       & 151     & 主压力最低限       \\
            73-75   & 变桨目标位置1-3      & 152     & 转子刹车压力最低限    \\
            76-78   & 变桨叶片1-3速度设定值   & 153     & 系统压力最大限      \\
            79-81   & 变桨叶片1-3加速设定值   & 154     & 需要解缆的角度值     \\
            82-84   & 变桨ref1-ref3位置值 & 155     & 偏航角度限值       \\
            85-87   & 叶片1-3位置        & 156-157 & 低风速一级、二级偏航角度 \\
            88-90   & 变桨叶片1-3位置冗余值   & 158-159 & 高风速一级、二级偏航角度 \\
            91-93   & 叶片1-3变桨速度      & 160-161 & 低风速一级、二级偏航延迟 \\
            94-96   & 变桨1-3电机电流      & 162-163 & 高风速一级、二级偏航延迟 \\
            97-99   & 变桨1-3伺服电机温度    & 164     & 解缆速度上限       \\
            100-102 & 变桨1-3温度        & 165     & 高风速阀值        \\
            103-105 & 变桨1-3电机温度      & 166     & 对风最小风速       \\
            106     & 变桨轮毂温度         & 167     & 偏航最小速度       \\
            107-109 & 变桨1-3柜顶盒温度     & 168     & 对风误差限值       \\
            110-112 & 变桨1-3柜底盒温度     & 169     & 齿轮箱输入轴温上限    \\
            113     & 桨距角给定值         & 170     & 齿轮箱输出轴温上限    \\
            114     & 控制器的转矩设定       & 171     & 齿轮箱油入口最大极限温度 \\
            115-117 & 变桨SG1故障码1-3    & 172     & 齿轮箱入口油温上限    \\
            118-120 & 变桨SG2故障码1-3    & 173     & 齿轮箱加热温度低限值   \\
            121-123 & 变桨变频器故障值1-3    & 174     & 启动水泵的油温限值    \\
            124-126 & 1-3变桨柜温度       & 175     & 启动风扇的油温限值    \\
            127-129 & 变桨1-3电容温度      & 176     & 启动高速油泵油温限值   \\
            130-132 & 变桨1-3电容电压      & 177     & 启动高速油泵速度限值   \\
            133-135 & 变桨1-3变频器温度     & 178     & 启动高速油泵速度限值   \\
            136-138 & 变桨1-3轮毂温度      & 179     & 润滑滤网压差上限     \\
            139     & 发电机转速          & 180     & 启动风扇的温度限值    \\
            140     & 扩展-转矩给定值       & 181     & 停止风扇的温度限值    \\
            141     & 变频器有功功率        & 182     & 进入运行状态的转速值   \\
            142     & 变频器无功功率        & 183     & 进入发电状态的转速    \\
            143     & 变频器控制转矩设定值     & 184     & 启动状态下的桨距角    \\
            144     & 发电机转矩          & 185     & 停机状态下的桨距角    \\
            145     & 叶轮转速给定         & 186     & 运行状态下的桨距角    \\
            146     & 无功给定           & 187     & 运行状态下的转速给定   \\
            147     & 液压泵电机反馈超时      & 188     & 发电状态下的转速给定   \\
            148     & 液压电机制动压力时间超时   & 189     & 启动机舱风扇的温度限值  \\
            149     & 液压泵启动压力        & 190     & 变桨系统初始化      \\
            150     & 液压泵停止压力        & 191     & 转矩限值         \\
            \bottomrule%第三道横线
            \end{tabular}}}
            \end{table}

    \section{附录2 Wavelet DenseNet-LSTM 算法源码}
    
    \lstinputlisting[
    style       =   Python,
    caption     =   {\bf Wavelet DenseNet-LSTM},
    label       =   {ff.py}
    ]{源码.py}

    \section{附录3 基于Matlab/Simulink的风电机组仿真框架 }

    \begin{figure}[H]
        \centering
        \includegraphics[width=18cm,height=15cm,angle=270]{毕设图片/附录-1}
        %\includegraphics[height=13.5cm,angle=270]{毕设图片/附录-1}
        \caption{海上风电机组整体仿真框架图}
        \label{fig:附录-1}
    \end{figure}

    \begin{figure}[H]
        \centering
        \includegraphics[width=16cm]{毕设图片/附录-2}
        \caption{齿轮箱仿真模型图}
        \label{fig:附录-2}
    \end{figure}

    \begin{figure}[H]
        \centering
        \includegraphics[width=16cm]{毕设图片/附录-3}
        \caption{传动系统仿真模型图}
        \label{fig:附录-3}
    \end{figure}

    \begin{figure}[H]
        \centering
        \includegraphics[width=16cm]{毕设图片/附录-4}
        \caption{双馈异步发电机仿真模型图}
        \label{fig:附录-4}
    \end{figure}

    \begin{figure}[H]
        \centering
        \includegraphics[width=16cm]{毕设图片/附录-5}
        \caption{电网仿真模型图}
        \label{fig:附录-5}
    \end{figure}

    \begin{figure}[H]
        \centering
        \includegraphics[width=16cm]{毕设图片/附录-6}
        \caption{基于CNN的虚拟传感器和执行器}
        \label{fig:附录-6}
    \end{figure}

}