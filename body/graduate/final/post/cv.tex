% \cleardoublepage
% \chapternonum{作者简介}

% {\noindent \textbf{一、个人信息}}

% \noindent 荣振威,男,汉族,中共党员,1999年出生于安徽省安庆市

% \vspace{0.6cm}
% {\noindent \textbf{一、教育经历}}

% \noindent 2021.09$ \sim $2024.06 浙江大学,机械专业,硕士研究生

% \noindent 2017.09$ \sim $2021.06 浙江工业大学,机械工程专业,本科

% \vspace{0.6cm}
% {\noindent \textbf{一、科研成果}}

% \begin{enumerate}[leftmargin=0.75cm, itemsep=0cm]
% \item [{[1]}]
% % Underwater Stereo Camera Refractive Calibration and Image Restoration Using Coordinate Constraints[J]. IEEE Robotics and Automation Letters.(一作,SCI,中科院二区,Under Review)
% 以第一作者在IEEE Robotics and Automation Letters上投稿论文一篇(Under Review)
% \item [{[2]}]
% % A resident subsea docking system with a real-time communication buoy moored by an electro-optical-mechanical cable[J]. Ocean Engineering, 2023, 271: 113729.(二作,SCI,中科院一区)
% 以第二作者在Ocean Engineering上发表论文一篇
% \item [{[3]}]
% % Unmanned surface vehicle collision avoidance path planning in restricted waters using multi-objective optimisation complying with colregs[J]. Sensors, 2022, 22(15): 5796.(二作,SCI,中科院三区)
% 以第二作者在Sensors上发表论文一篇
% \item [{[4]}]
% % 一种基于结构光的水下高精度三维成像装置及方法. CN116817794B, 2024.(发明专利,除导师外第一发明人,已授权)
% 以第二作者授权一项国家发明专利
% \end{enumerate}

% \vspace{0.6cm}
% {\noindent \textbf{二、项目经历}}
% \begin{enumerate}[leftmargin=0.75cm, itemsep=0cm]
%     \item [{[1]}]
%     机器人技术国家重点实验室开放基金:海上风电水下运维机器人自主接驳控制技术研究,学生骨干。
%     \item [{[2]}]
%     国家自然科学基金国际交流合作项目:基于无人系统和人工智能的海上风力机健康监测与容错控制研究,国家自然科学基金国际交流合作项目,学生骨干。
%     \item [{[3]}]
%     国家重点研发计划深海关键技术与装备专项项目:水下直升机,学生骨干。    
% \end{enumerate}

% \vspace{0.6cm}
% {\noindent \textbf{三、在校荣誉}}

% \begin{enumerate}[leftmargin=0.75cm, itemsep=0cm]
%     \item [{[1]}]
%     2023年 第十八届“挑战杯”全国大学生课外学术科技作品竞赛全国二等奖
%     \item [{[2]}]
%     2023年 第十一届全国航行器设计与制作大赛全国二等奖
%     \item [{[3]}]
%     2024年 浙江大学优秀毕业生
%     \item [{[4]}]
%     2023年 浙江大学优秀研究生
%     \item [{[5]}]
%     2022年 浙江大学优秀研究生
% \end{enumerate}



\cleardoublepage
\chapternonum{作者简介}

{\noindent \textbf{一、个人信息}}

\noindent 荣振威,男,汉族,中共党员,1999年出生于安徽省安庆市

\vspace{0.6cm}
{\noindent \textbf{二、教育经历}}

\noindent 2021.09$ \sim $2024.06 浙江大学,机械专业,硕士研究生

\noindent 2017.09$ \sim $2021.06 浙江工业大学,机械工程专业,本科

\vspace{0.6cm}
{\noindent \textbf{三、科研成果}}

\begin{enumerate}[leftmargin=0.75cm, itemsep=0cm]
\item [{[1]}]
\textbf{Rong Z}, Wei H, Cai C, et al. Underwater Stereo Camera Refractive Calibration and Image Restoration Using Coordinate Constraints[J]. IEEE Robotics and Automation Letters.(一作,SCI,中科院二区,Under Review)
\item [{[2]}]
Cai C, \textbf{Rong Z}, Chen Z, et al. A resident subsea docking system with a real-time communication buoy moored by an electro-optical-mechanical cable[J]. Ocean Engineering, 2023, 271: 113729.(二作,SCI,中科院一区)
\item [{[3]}]
Gu Y, \textbf{Rong Z}, Tong H, et al. Unmanned surface vehicle collision avoidance path planning in restricted waters using multi-objective optimisation complying with colregs[J]. Sensors, 2022, 22(15): 5796.(二作,SCI,中科院三区)
\item [{[4]}]
司玉林, \textbf{荣振威}, 顾阳, 等. 一种基于结构光的水下高精度三维成像装置及方法. CN116817794B, 2024.(发明专利,除导师外第一发明人,已授权)
\end{enumerate}

\vspace{0.6cm}
{\noindent \textbf{四、项目经历}}
\begin{enumerate}[leftmargin=0.75cm, itemsep=0cm]
    \item [{[1]}]
    海上风电水下运维机器人自主接驳控制技术研究,机器人技术国家重点试验室开放基金。学生骨干。
    \item [{[2]}]
    基于无人系统和人工智能的海上风力机健康监测与容错控制研究,国家自然科学基金国际交流合作项目。学生骨干。
    \item [{[3]}]
    水下直升机,国家重点研发计划。学生骨干。    
\end{enumerate}

\vspace{0.6cm}
{\noindent \textbf{五、在校荣誉}}

\begin{enumerate}[leftmargin=0.75cm, itemsep=0cm]
    \item [{[1]}]
    2023.09 第十八届“挑战杯”全国大学生课外学术科技作品竞赛全国二等奖
    \item [{[2]}]
    2023.09 第十一届全国航行器设计与制作大赛全国二等奖
    \item [{[3]}]
    2024.03 浙江大学优秀毕业生
    \item [{[4]}]
    2023.09 浙江大学优秀研究生
    \item [{[5]}]
    2022.09 浙江大学优秀研究生
\end{enumerate}

